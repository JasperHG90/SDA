\documentclass[]{article}
\usepackage{lmodern}
\usepackage{amssymb,amsmath}
\usepackage{ifxetex,ifluatex}
\usepackage{fixltx2e} % provides \textsubscript
\ifnum 0\ifxetex 1\fi\ifluatex 1\fi=0 % if pdftex
  \usepackage[T1]{fontenc}
  \usepackage[utf8]{inputenc}
\else % if luatex or xelatex
  \ifxetex
    \usepackage{mathspec}
  \else
    \usepackage{fontspec}
  \fi
  \defaultfontfeatures{Ligatures=TeX,Scale=MatchLowercase}
\fi
% use upquote if available, for straight quotes in verbatim environments
\IfFileExists{upquote.sty}{\usepackage{upquote}}{}
% use microtype if available
\IfFileExists{microtype.sty}{%
\usepackage{microtype}
\UseMicrotypeSet[protrusion]{basicmath} % disable protrusion for tt fonts
}{}
\usepackage[margin=1in]{geometry}
\usepackage{hyperref}
\hypersetup{unicode=true,
            pdftitle={Gender pay gap notes},
            pdfborder={0 0 0},
            breaklinks=true}
\urlstyle{same}  % don't use monospace font for urls
\usepackage{color}
\usepackage{fancyvrb}
\newcommand{\VerbBar}{|}
\newcommand{\VERB}{\Verb[commandchars=\\\{\}]}
\DefineVerbatimEnvironment{Highlighting}{Verbatim}{commandchars=\\\{\}}
% Add ',fontsize=\small' for more characters per line
\usepackage{framed}
\definecolor{shadecolor}{RGB}{248,248,248}
\newenvironment{Shaded}{\begin{snugshade}}{\end{snugshade}}
\newcommand{\KeywordTok}[1]{\textcolor[rgb]{0.13,0.29,0.53}{\textbf{#1}}}
\newcommand{\DataTypeTok}[1]{\textcolor[rgb]{0.13,0.29,0.53}{#1}}
\newcommand{\DecValTok}[1]{\textcolor[rgb]{0.00,0.00,0.81}{#1}}
\newcommand{\BaseNTok}[1]{\textcolor[rgb]{0.00,0.00,0.81}{#1}}
\newcommand{\FloatTok}[1]{\textcolor[rgb]{0.00,0.00,0.81}{#1}}
\newcommand{\ConstantTok}[1]{\textcolor[rgb]{0.00,0.00,0.00}{#1}}
\newcommand{\CharTok}[1]{\textcolor[rgb]{0.31,0.60,0.02}{#1}}
\newcommand{\SpecialCharTok}[1]{\textcolor[rgb]{0.00,0.00,0.00}{#1}}
\newcommand{\StringTok}[1]{\textcolor[rgb]{0.31,0.60,0.02}{#1}}
\newcommand{\VerbatimStringTok}[1]{\textcolor[rgb]{0.31,0.60,0.02}{#1}}
\newcommand{\SpecialStringTok}[1]{\textcolor[rgb]{0.31,0.60,0.02}{#1}}
\newcommand{\ImportTok}[1]{#1}
\newcommand{\CommentTok}[1]{\textcolor[rgb]{0.56,0.35,0.01}{\textit{#1}}}
\newcommand{\DocumentationTok}[1]{\textcolor[rgb]{0.56,0.35,0.01}{\textbf{\textit{#1}}}}
\newcommand{\AnnotationTok}[1]{\textcolor[rgb]{0.56,0.35,0.01}{\textbf{\textit{#1}}}}
\newcommand{\CommentVarTok}[1]{\textcolor[rgb]{0.56,0.35,0.01}{\textbf{\textit{#1}}}}
\newcommand{\OtherTok}[1]{\textcolor[rgb]{0.56,0.35,0.01}{#1}}
\newcommand{\FunctionTok}[1]{\textcolor[rgb]{0.00,0.00,0.00}{#1}}
\newcommand{\VariableTok}[1]{\textcolor[rgb]{0.00,0.00,0.00}{#1}}
\newcommand{\ControlFlowTok}[1]{\textcolor[rgb]{0.13,0.29,0.53}{\textbf{#1}}}
\newcommand{\OperatorTok}[1]{\textcolor[rgb]{0.81,0.36,0.00}{\textbf{#1}}}
\newcommand{\BuiltInTok}[1]{#1}
\newcommand{\ExtensionTok}[1]{#1}
\newcommand{\PreprocessorTok}[1]{\textcolor[rgb]{0.56,0.35,0.01}{\textit{#1}}}
\newcommand{\AttributeTok}[1]{\textcolor[rgb]{0.77,0.63,0.00}{#1}}
\newcommand{\RegionMarkerTok}[1]{#1}
\newcommand{\InformationTok}[1]{\textcolor[rgb]{0.56,0.35,0.01}{\textbf{\textit{#1}}}}
\newcommand{\WarningTok}[1]{\textcolor[rgb]{0.56,0.35,0.01}{\textbf{\textit{#1}}}}
\newcommand{\AlertTok}[1]{\textcolor[rgb]{0.94,0.16,0.16}{#1}}
\newcommand{\ErrorTok}[1]{\textcolor[rgb]{0.64,0.00,0.00}{\textbf{#1}}}
\newcommand{\NormalTok}[1]{#1}
\usepackage{graphicx,grffile}
\makeatletter
\def\maxwidth{\ifdim\Gin@nat@width>\linewidth\linewidth\else\Gin@nat@width\fi}
\def\maxheight{\ifdim\Gin@nat@height>\textheight\textheight\else\Gin@nat@height\fi}
\makeatother
% Scale images if necessary, so that they will not overflow the page
% margins by default, and it is still possible to overwrite the defaults
% using explicit options in \includegraphics[width, height, ...]{}
\setkeys{Gin}{width=\maxwidth,height=\maxheight,keepaspectratio}
\IfFileExists{parskip.sty}{%
\usepackage{parskip}
}{% else
\setlength{\parindent}{0pt}
\setlength{\parskip}{6pt plus 2pt minus 1pt}
}
\setlength{\emergencystretch}{3em}  % prevent overfull lines
\providecommand{\tightlist}{%
  \setlength{\itemsep}{0pt}\setlength{\parskip}{0pt}}
\setcounter{secnumdepth}{0}
% Redefines (sub)paragraphs to behave more like sections
\ifx\paragraph\undefined\else
\let\oldparagraph\paragraph
\renewcommand{\paragraph}[1]{\oldparagraph{#1}\mbox{}}
\fi
\ifx\subparagraph\undefined\else
\let\oldsubparagraph\subparagraph
\renewcommand{\subparagraph}[1]{\oldsubparagraph{#1}\mbox{}}
\fi

%%% Use protect on footnotes to avoid problems with footnotes in titles
\let\rmarkdownfootnote\footnote%
\def\footnote{\protect\rmarkdownfootnote}

%%% Change title format to be more compact
\usepackage{titling}

% Create subtitle command for use in maketitle
\newcommand{\subtitle}[1]{
  \posttitle{
    \begin{center}\large#1\end{center}
    }
}

\setlength{\droptitle}{-2em}

  \title{Gender pay gap notes}
    \pretitle{\vspace{\droptitle}\centering\huge}
  \posttitle{\par}
    \author{}
    \preauthor{}\postauthor{}
    \date{}
    \predate{}\postdate{}
  

\begin{document}
\maketitle

The Gender Pay Gap defines the difference in hourly wage between men and
women. This can be measured either by the mean wage, the median wage, or
any other average. It is of interest not only to assess the size of the
gender gap across a population, but also to uncover the driving factors
behind its existence. For instance, are certain high-wage industries
dominated by male workers? Does the population wage gap increase in
positions of high wage? Do men and women earn comparable salaries for
comparable jobs?

\begin{Shaded}
\begin{Highlighting}[]
\NormalTok{i <-}\StringTok{ }\KeywordTok{seq}\NormalTok{(}\DecValTok{0}\NormalTok{,}\DecValTok{5}\NormalTok{,}\DecValTok{1}\NormalTok{)}
\KeywordTok{print}\NormalTok{(i)}
\end{Highlighting}
\end{Shaded}

\begin{verbatim}
## [1] 0 1 2 3 4 5
\end{verbatim}

This report will focus on the wage gap in the UK in 2017. The data we
analyse throughout the report comprise all companies with over 250
employees, and a small amount of smaller companies. As of 2017, it is
required by the UK government that all such companies publish
information pertaining to the pay of employees for the purpose of
showing whether there is indeed a difference in the pay of male and
female employees. This dataset is well protected against non-response
errors, given that failure to comply with the regulations can lead to
enforcement from the Equality and Human Rights Commission {[}1{]}. In
addition, non-sampling errors are also well protected via the use of
clear and simple categories, for example, size brackets for the number
of employees. Our aim is to create an efficient sampling design, from
which we can make accurate calculations by using a smaller sample. We
will compare the precision of a number of different such designs. Their
accuracy will be measured against the results obtained from the complete
data analysis.

One element of the data which we will refer to throughout this report is
the Standard Industrial Classification (SIC). This is a widely used
system for classifying companies into certain larger industries. Another
important variable in the data is the EmployerSize, which categorises
each company into seven distinct sizes: size not provided, less than
250, 250 to 499, 500 to 999, 1000 to 4999, 5000 to 19,999, and 20,000 or
more. We will discuss in detail whether using information on the gender
pay gap within these categories can lead to better sampling designs.

Before conducting our analysis, we first made some modifications to the
dataset. Firstly, the information that was not necessary was stored in a
separate dataframe, which we did not use. For example, the person
responsible for the submission of the data was not relevant, and was
thus not included in our main dataset.

\section{Population data analysis}\label{population-data-analysis}

Our analysis found that, across the population (of provided data), the
average monthly wage for men was 14.4\% higher than that for women. It
should be noted that this provides only an initial insight into the
extent of the gender gap in the UK, and should be interpreted with
caution. For instance, it can often be more relevant to assess the size
of gender pay gaps in terms of the median wage, so that the few
disproportionately high earners do not skew the results. To improve the
accuracy of this `population' gender gap we should have weighted the
population mean based on the size of the companies, so that the larger
companies hold more influence on the population statistics. However,
with the data provided, it would be difficult to achieve a meaningful
accuracy, given that we are not provided with the exact size of the
companies.

A brief analysis into the prevalence of the gender gap, namely in how
many companies there exists a disparity in wages based on gender,
revealed the following: 749, or 11.15\%, of the companies in the dataset
exhibited a gender gap (in the average hourly rate) favouring women,
whereas 5910 companies, or 88.03\%, paid men a higher average wage. (The
remaining 0.008\% of companies exhibited no gender wage gap). This
indicates that many of the industry-wide differences in pay are likely
to favour men. These figures are accompanied by the histogram (Figure
XX), showing a heavy skew towards a `positive' pay gap. Recall that a
positive pay gap represents one in which males receive a higher wage
than females.

The gender pay gap in the UK is also evident in the distribution of
bonus payments; defined as any payment additional to an employee's
salary. The gender gap for bonus payments stands at 14.1\% at the UK
population level, again in favour of men. As previously, this statistic
falls short of a true population average, owing to the lack of
appropriate weighting. However, for the purpose of brevity, these will
be referred to herein as the true population values. As previously
mentioned, comparing the median (hourly) wage within a company can be a
more effective way to study the gender pay gap. Therefore, by averaging
the median wage across all UK companies, we found that the median male
employee earns 12.2\% more than his female counterpart.

A glance at the distribution of both positive and negative wage
inequalities again proves interesting: 903, or 13.45\%, of UK companies
pay a higher median wage to women in comparison to men - a significantly
smaller figure than the 5258 companies, or 78.32\%, who provide a higher
median wage to males. (The remaining 552 companies, or 8.22\%, provide
an equal median wage). Similarly to the analysis of the mean wage, we
can see from the histogram (Figure XX) that the distribution is strongly
skewed in favour of higher male income.

Using the data on the gender of employees working within each income
quartile (herein referred to as lower, lower middle, upper middle, top),
we were able to get a sense of the distribution of employees at the
company level. For example Figure XX shows that a higher concentration
of women currently work in the lower quartile, while male employees
dominate the top and upper middle quartiles. This is likely to be one of
the driving factors behind the 14.4\% nation-wide pay inequality.
However, in order to further this analysis, we would require data
pertaining to the size of the gender pay gap within each income
quartile. For example, we know that males dominate the top quartile -
and females the lower quartile - but we do not know in which quartile is
the pay gap more pronounced.

\section{Sampling methods}\label{sampling-methods}

The focus of this report largely falls on sampling methods, with the aim
of making inferences on the target population using a (smaller) designed
sample. The target population for our sampling is the UK working
population; the goal of assessing the gender wage gap is to do so in the
context of its widespread presence in the entire population. In the
current situation, whereby companies with over 250 employees are
required to provide relevant data, we can work with a dataset containing
6713 companies. The sample size in mind when conducting the following
sample analyses will be 1000. Thus, we endeavour to make calculations
related to the gender pay gap on such a sample which accurately reflect
the UK population.

Firstly, we opted to generate a sample of 1000 random companies out of
the 6713 available, with each company given an equal probability of
being sampled (1000 / 6713 = 14.9\%). This method is known as a simple
random sample (SRS), and requires the assumption that a random sample is
representative of the target population. The sample is taken without
replacement, i.e.~no company can be included more than once in the
sample.

Working with this SRS of 1000 companies, we found that the gender pay
gap stands at 12.56\% and 14.88\% when assessing the median and mean
wage, respectively. Both of these pay gaps are in favour of men. These
values do not differ drastically from the population values of 12.2\%
and 14.1\%. 95\% confidence intervals for these percentages (.. , ..)
show little variation from our original population values. The standard
error for the estimate for the difference in mean hourly wage using the
SRS is 0.0041. Using this sampling design, we achieve a margin of error
of 0.0041*1.96 = 0.8\% at the 5\% level. This relates to the sampling
error - the difference between the sampled statistic and the population
statistic is just 0.8\%. We will use this SRS as a reference to compare
all future sampling designs against. This will be done via the design
effect (Variance\_design / Variance\_SRS).

When the statistic of interest (here, gender wage inequality) is known
to correlate in some way with information provided in the sampling
frame, it can be useful to stratify. Arranging the data into strata
prior to sampling can optimise the sampling design in the sense that the
standard error of the `predictions' can be reduced, thus improving the
precision. This will only be the case if we see small within-strata
variance, and larger between-strata variance. This will make for a more
efficient design for the reason that, if a large portion of the total
variance is between the strata, only a small sample from each stratum
will be necessary to make accurate predictions.

Based on rudimentary analysis of the two potential stratification
variables SIC and EmployerSize, and with the sample of 1000 still the
goal, we decided to stratify based on SIC division. Evidence supporting
this decision is provided in Figure XX: violin plots of the strata. From
these plots we can clearly rule out EmployerSize as a useful
stratification variable given the great variation in wage difference
within each employer size category. Additionally, post-hoc tests
following an ANOVA on the different categories of EmployerSize revealed
very little evidence of between-group differences (only one of the 21
pairwise comparisons showed a significant difference). On the contrary,
running an ANOVA on the SIC divisions yielded many significant pairwise
differences in the median pay gap.

SIC division was identified as a more useful stratification variable.
However, there was yet an obvious problem in that two of the divisions
comprise just three companies each: activities of extraterritorial
organisations and bodies, and activities of households as employers. We
excluded these two divisions for the following reasons: 1. They comprise
just 6 companies, making it difficult to sample in a representative way.
2. They contain less than 0.09\% of population. 3. The divisions
themselves have unique, specific characteristics. Therefore, it is not
possible to combine them with other divisions in a meaningful way.

We then used a proportional-to-size (PPS) procedure to draw a stratified
sample of (at most) 1000 companies. This technique ensures that the
drawn sample well represents the population, given that the proportions
of the divisions within the sample match those within the population.
For example, the division `education' contains 986 companies, or 14.69\%
of the population. Within the PPS sample, education will comprise as
close to this figure as possible. This method of sampling should ensure
lower sampling errors, when comparing with an SRS. This is because a PPS
stratified sample is expected to better represent the target population
(in this case all UK companies with over 250 employees).

The PPS sample tells us that there is a 14.24\% difference in the
average wage - closer to the population value (14.4\%) than the SRS
(14.88\%). The margin of error here is 0.74\% (standard error =
0.0037645), which gives a sense of how far our predictions of the wage
difference differ from the population values. This is calculated by
1.96*(s.e.), where 1.96 is the 95\% z-value used to construct confidence
intervals. The margin of error is, as expected, slightly lower than that
of the SRS (0.8\%). The design effect of 0.8771 means a sample of only
877 would be required to achieve the same precision as our earlier SRS.
These figures indicate that we can more efficiently (design effect) and
accurately (margin of error) make inferences on the population with this
sampling design.

However, it is still possible to optimise the sample, via the choice of
stratum sample size. This can be done by using a Neyman Allocation. This
method allocates an optimal size for each stratum via the formula (N\_h
* V\_h) / sum(N\_h * V\_h), where N\_h and V\_h are the stratum sizes
and variances respectively. In this way, it is possible to oversample
from certain strata - namely those with higher variances - in order to
achieve a smaller standard error. According to this sampling design, the
gender pay gap stands at 14.05\%. While this underestimates the pay gap
slightly more than the PPS sample, the standard error of 0.0037412 is
marginally smaller in comparison. This means we now have a smaller,
better design effect of 0.8685. Consequently, we consider this design
more efficient than the PPS, requiring fewer companies to achieve the
same precision. We therefore conclude that, when seeking to maximise
precision within a sample of 1000 companies, stratifying based on SIC
division and using Neyman's optimal allocation is better than taking an
SRS or a PPS stratified sample. Other sampling designs are beyond the
scope of this report.

\section{Sample size determination based on the coefficient of
variation}\label{sample-size-determination-based-on-the-coefficient-of-variation}

Another common way in which to sample is to base the sample size around
a prerequired level of precision. In this situation, the precision of
the sample estimates dictates the sample size, as opposed to a fixed
budget or sample size. With this technique in mind, we aim to find the
sample size required in order to achieve a coefficient of variation of
0.01. The coefficient of variation (CV) is defined as the standard error
of the wage gap, relative to the size of the wage gap - a standardised
error term.

Firstly, we calculated the existing CV, taken from the optimised
stratified sample, to be 0.026. This is significantly higher than the
target of 0.01. In order to calculate the optimal sample size in this
situation, we can systematically increase the sample size, n, until the
CV falls at or below the target of 0.01. This algorithm is implemented
in R by the use of for loops and if statements. By doing so, we find
that a sample of 3919 is sufficient. Therefore, with a level of
precision as determined by a CV of 0.01, it would be necessary only to
sample 58\% of UK companies with over 250 employees, while still
obtaining the same information.

It is also of great interest to sample at the level of the individual
employee, rather than at the company level. This can be done with a
cluster sampling design. This method requires the weighting of the data
according to the size of the company. In absence of the exact company
size data, we opted to use the midpoints of the size categories as
surrogate company sizes. The largest size category was an exception; we
used the value of 20,000 to weight these companies. Companies not
providing the number of employees were weighted as 0, and thus excluded
from this analysis. Using this sampling design, we found that a weighted
average for the gender pay gap stands at 16\%, in favour of men.

Discuss potential non-sampling, sampling, and non-response errors.

\section{8. Another way to think about sampling is not at the company
level, but at the level of the individual. Unequal pay at very large
companies of course has a larger impact than at small companies. We lack
data for companies with \textless{}250 employees, but could still weight
the data according to the size of the company to inspect the gender pay
gap for individuals in the UK who work at companies with \textgreater{}
250
employees.}\label{another-way-to-think-about-sampling-is-not-at-the-company-level-but-at-the-level-of-the-individual.-unequal-pay-at-very-large-companies-of-course-has-a-larger-impact-than-at-small-companies.-we-lack-data-for-companies-with-250-employees-but-could-still-weight-the-data-according-to-the-size-of-the-company-to-inspect-the-gender-pay-gap-for-individuals-in-the-uk-who-work-at-companies-with-250-employees.}

\section{a. Describe what you would do if the goal was not to study the
gender pay gap at companies, but for individuals at
companies.}\label{a.-describe-what-you-would-do-if-the-goal-was-not-to-study-the-gender-pay-gap-at-companies-but-for-individuals-at-companies.}

\section{b. Use the strategy you describe under a. to finally say
something about the size of the gender pay
gap.}\label{b.-use-the-strategy-you-describe-under-a.-to-finally-say-something-about-the-size-of-the-gender-pay-gap.}

References: {[}1{]} BRIEFING PAPER Number 7068, 28 September 2018 The
Gender Pay Gap www.parliament.uk/commons-library \textbar{}
intranet.parliament.uk/commons-library \textbar{}
\href{mailto:papers@parliament.uk}{\nolinkurl{papers@parliament.uk}}
\textbar{} @commonslibrary By Feargal McGuinness Doug Pyper \#\# TODO:

\begin{itemize}
\tightlist
\item
  Stratification:
\item
  Compare the two variables
\item
  Recode the SIC variables
\item
  Horvitz-thompson (Neyman) optimal allocation
\item
  How to oversample from specific strata to reduce overall variance (See
  code Peter week 5 on stratification)
\end{itemize}

throwaway work: \#show the violin plot here to show that the gap differs
among size groups (the reason to stratify here) Another variable on
which to stratify is EmployerSize. The data are already provided in
seven categories based on the number of employees. Therefore, by using
these as the strata, we can improve both the precision of the estimates
within each strata, and thus the overall precision. We could not sample
an equal number (1000 / 7) of companies from each strata, owing to there
not being a sufficient number of companies in each strata. Therefore, we
sampled from each size division with equal probability. (CHECK! THIS WAS
CONFUSING). (We have roughly taken a proportion of 1000/6713 = 14.89\%
from each size division). In order to do this, it was first necessary to
calculate the proportion of all companies falling in each size division.
This proportion was then multiplied by 1000 for each strata to achieve
the desired sample size. Given that we preferred our sample to be at
most 1000 companies, we rounded the final stratum sizes down where
necessary. \#report SE and confidence intervals, talk about the features
(errors) of this particular design

Given that the premise of stratified samples is to increase the
precision/ lower the SE, this sample has not performed its desired
function. Therefore, it was necessary for us to implement an optimal
allocation strategy, so that each size division is more apporopriately
represented in the sample. It is of interest to take a larger sample
from strata in which there is more variation. For this, we (attempted to
use) used the Neyman Allocation. \ldots{}


\end{document}
